\section*{Теоретическое введение}

\subsection*{Прямое и обратное преобразование Фурье}

Рассмотрим физическую систему, на вход которой подаётся меняющийся со временем сигнал $f(t)$. Поставим задачу определить реакцию $g(t)$ этой системы на входное воздействие, называемую \textit{откликом}:
$$
f(t) \rightarrow \boxed{\sigoper} \rightarrow g(t).
$$
Для удобства дальнейшего изложения введём оператор $\sigoper$, который преобразует входной сигнал $f(t)$ в выходной $g(t)$:
$$
g = \sigoper(f)
$$
В общем случае нахождение отклика системы для произвольного сигнала затруднительно. Если система является линейной, то для любых входных сигналов $f_1(t)$ и $f_2(t)$ и любых чисел $\lambda_1, \lambda_2 \in \mathbb{R}$ выполнено:
$$
\sigoper(\lambda_1 \cdot f_1 + \lambda_2 \cdot f_2) = \lambda_1 \cdot \sigoper(f_1) + \lambda_2 \cdot \sigoper(f_2)
$$
Таким образом, решение поставленной задачи упрощается. Достаточно разложить входной сигнал в линейную комбинацию более простых составляющих, и исследовать отклик системы на каждой из них.

Пусть входной сигнал является периодическим с периодом $T$. Тогда согласно теоремы Фурье его можно представить в виде суперпозиции синусоид с периодами $T_0, 2T_0, 3T_0, \dots$ или частотами $\omega_n = n \omega_0$. В комплексной форме ряд записывается в виде:
$$
f(t) = \sum_{n = 0}^{\infty}{a_n \cos{(\omega_n t + \varphi_n)}}
$$
Данное представление $f(t)$ называется \textit{рядом Фурье}, отдельные слагаемые $a_n \cos{(\omega_n t + \varphi_n)}$ называются \textit{гармониками}, а совокупность гармоник --- \textit{спектром}.

Удобно записать ряд Фурье в комплексной форме:
$$
f(t) = \sum_{n = -\infty}^{\infty}{c_n e^{i w_n t}}
$$
При этом вводятся формальные отрицательные частоты. Докажем эквивалентность записи ряда Фурье в комплексной и вещественной формах, для этого найдём связь между $a_n$ и $c_n$. Воспользуемся формулой Эйлера:
$$
e^{i (w_n t + \varphi_n)} =\cos{(w_n t + \varphi_n)} + i \sin{(w_n t + \varphi_n)} = e^{i \varphi_n} e^{i w_n t}
$$
$$
a_n \cos{(w_n t + \varphi_n)} = \frac{a_n}{2} e^{i \varphi_n} e^{i w_n t} + \frac{a_n}{2} e^{-i \varphi_n} e^{-i w_n t}
$$
Чтобы представление в вещественной и комплексной форме совпадали, необходимо и достаточно, чтобы:
\begin{equation*}
	\begin{cases}
		c_n = \frac{a_n}{2} e^{i \varphi_n} \\
		c_{-n} = \frac{a_n}{2} e^{-i \varphi_n} = \overline{c_n}
	\end{cases}
\end{equation*}

% TODO: формула прмого преобразования

% TODO: непериодический процесс

% TODO: соотношение неопределенностей

% TODO: физический смысл спектрального разложения

% TODO: амплитудно-модулированный сигнал

% TODO: сигнал модулированный по фазе

% TODO: картинки.