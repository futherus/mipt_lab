\documentclass[12pt,a4paper]{article}
\usepackage[T2A]{fontenc}
\usepackage[utf8]{inputenc}
\usepackage[russian]{babel}
\usepackage{amsmath}
\usepackage{amssymb}
\usepackage{graphicx}
\usepackage{floatrow}
\usepackage{booktabs}
%\usepackage{wrapfig}
\usepackage{lipsum}
%\usepackage{subcaption}
\usepackage{fancyhdr}


%multi-column
%\multicolumn{number cols}{align}{text} % align: l,c,r

%multi-row
\usepackage{multirow}

\newcommand{\figref}[1]{(См. рис. \ref{#1})}
\newcommand{\secref}[1]{(См. раздел. \ref{#1})}

\newcommand{\e}[1]{\text{$\cdot10^{#1}$}}

\pagestyle{fancy}
\fancyhead{}
\fancyhead[L]{Работа 2.3.1}
\fancyhead[R]{}
\fancyfoot[C]{\thepage}

\renewcommand{\cot}{\text{ctg}}

\author{\normalsize Выполнил: Дедков Денис, группа Б01-109 \\
	\normalsize 01.10.2022}
\date{}

\usepackage{float}
\restylefloat{table}
\title{
	\large Отчет о выполнении лабораторной работы 3.6.1 \\
	\Large Спектральный анализ электрических сигналов \\ 
	
}

\begin{document}
\maketitle
\subsection*{Цель работы} Изучить спектральный состав периодических электрических сигналов.

\subsection*{Оборудование и приборы} Генератор сигналов специальной формы
АКИП-3409/4, Цифровой осциллограф SIGLENT АКИП 4131/1.

\subsection*{Введение}
В работе изучается спектральный состав периодических электрических сигналов различной формы: последовательности прямоугольных
импульсов, последовательности цугов и амплитудно-модулированных
гармонических колебаний. Спектры этих сигналов наблюдаются с помощью анализатора спектра и сравниваются c рассчитанными теоретически.

В последнее время повсеместное распространение получила цифровая обработка сигналов. Спектральный состав оцифрованного сигнала
может быть найден численно. Существуют алгоритмы (быстрое преобразование Фурье, FFT), позволяющие проводить вычисления коэффициентов Фурье в реальном времени для сигналов относительно высокой
частоты (до 200 МГц). Именно быстрое преобразование Фурье используется в данной работе для вычисления спектра.
\newpage
	
\subsection*{Ход работы}
\subsubsection*{ Исследование спектра периодической последовательности прямоугольных импульсов}
Теоретическое описание спектра периодической последовательности прямоугольных импульсов приведено на рисунке \ref{res:impulse}.

\begin{figure}[H]
	\centering
	\begin{minipage}[b]{.49\textwidth}
		\centering
		\includegraphics[width=0.9\linewidth]{"res/impulse"}
		\caption{Периодическая последовательность импульсов и её спектр.}
		\label{res:impulse}
	\end{minipage}%
	\begin{minipage}[b]{.49\textwidth}
		\centering
		\includegraphics[width=0.9\linewidth]{"res/impulse_spectrum"}
		\label{res:impulse_spectrum}
	\end{minipage}
\end{figure}

Настраиваем генератор на прямоугольные импульсы с частотой повторения $\nu_{\text{повт}} = 1 \text{кГц}$ (период $T = 1 \text{мс}$) и длительностью импульса $\tau = \frac{T}{20} = 50 \text{мкс}$. Фотографии экрана электронного осциллографа приведены на рисунках \ref{photo:impulse_nu} и \ref{photo:impulse_tau}. 

\begin{figure}[H]
	\centering
	\begin{minipage}[b]{.5\textwidth}
		\centering
		\includegraphics[width=0.9\linewidth]{"photo/impulse1"}
	\end{minipage}%
	\begin{minipage}[b]{.5\textwidth}
		\centering
		\includegraphics[width=0.9\linewidth]{"photo/impulse2"}
	\end{minipage}
\end{figure}

\begin{figure}[H]
	\centering
	\begin{minipage}[b]{.5\textwidth}
		\centering
		\includegraphics[width=0.9\linewidth]{"photo/impulse3"}
		\label{photo:impulse_nu}
		\caption{Изменения спектра при увеличении $\nu$}
	\end{minipage}%
	\begin{minipage}[b]{.5\textwidth}
		\centering
		\includegraphics[width=0.9\linewidth]{"photo/impulse4"}
	\end{minipage}
\end{figure}


\begin{figure}[H]
	\centering
	\begin{minipage}[b]{.33\textwidth}
		\centering
		\includegraphics[width=0.9\linewidth]{"photo/impulse5"}
		\label{photo:impulse_tau}
		\caption{Изменения спектра при увеличении $\tau$}
	\end{minipage}%
	\begin{minipage}[b]{.33\textwidth}
		\centering
		\includegraphics[width=0.9\linewidth]{"photo/impulse6"}
	\end{minipage}
	\begin{minipage}[b]{.33\textwidth}
		\centering
		\includegraphics[width=0.9\linewidth]{"photo/impulse7"}
	\end{minipage}
\end{figure}

При фиксированных параметрах $\nu_{\text{повт}} = 1 \text{кГц}$ и $\tau = 100 \text{мкс}$ измерим высоты (амплитуды) $a_n$ и частоты $\nu_n$ несколько первых гармоник спектра и сравним их значения с рассчитанными теоретически по следующей формуле:
$$a_n = \frac{\sin{(\pi n \tau/T)}}{\pi n}$$

Графики экспериментально измеренного спектра, а также теоретически рассчитанной огибающей приведен на рисунке \ref{fig:a9}.

\noindent \begin{minipage}[B]{.5\textwidth}
\begin{figure}[H]
	\centering
	\includegraphics[width=0.8\linewidth]{"gen/fig-a9.pdf"}
	\label{fig:a9}
	\caption{Спектр и огибающая}
\end{figure}
\end{minipage}%
\begin{minipage}[B]{.5\textwidth}
\begin{figure}[H]
	\centering
	\includegraphics[width=0.7\linewidth]{"photo/impulse5"}
	\label{photo:impulse_tau}
	\caption{Фотография экспериментального спектра}
\end{figure}
\end{minipage}

Проводим измерения зависимости ширины спектра $\Delta\nu$ от времени
импульса $\tau$ в диапазоне от $20$ до $200$ мкс при фиксированной $\nu_{\text{повт}}$ (ширина
измеряется от центра спектра до первой нулевой гармоники). Строим график зависимости $\Delta\nu(\frac{1}{\tau})$ (см. рис. \ref{fig:a11}).

\noindent \begin{minipage}[B]{.5\textwidth}
	\begin{figure}[H]
		\centering
		\includegraphics[width=1.2\linewidth]{"gen/fig-a11.pdf"}
		\label{fig:a11}
		\caption{График зависимости $\Delta\nu(\frac{1}{\tau})$ }
	\end{figure}
\end{minipage}%
\begin{minipage}[B]{.5\textwidth}
	\begin{table}[H]
		\footnotesize
		\input{gen/tab-a11.tex}
		\caption{Данные}
		\label{tab:a11}
	\end{table}
\end{minipage}

\begin{table}[H]
	\footnotesize
	\input{gen/tab-a11-mnk.tex}
	\caption{Обработка МНК}
	\label{tab:a11-mnk}
\end{table}

Погрешность эксперимента коррелирует с погрешностью коэффициента наклона. Следовательно, рассчитаем количественный критерий точности:

$$\mathcal{C} \approx \frac{\Delta a}{a} \approx 0.5 \% $$


\subsubsection*{Исследование спектра периодической последовательности цугов}

Теоретическое описание спектра периодической последовательности цугов приведено на рисунке \ref{res:zug}.

\begin{figure}[H]
	\centering
	\begin{minipage}[b]{.49\textwidth}
		\centering
		\includegraphics[width=0.9\linewidth]{"res/zug"}
		\caption{Периодическая последовательность импульсов и её спектр.}
		\label{res:zug}
	\end{minipage}%
	\begin{minipage}[b]{.49\textwidth}
		\centering
		\includegraphics[width=0.9\linewidth]{"res/zug_spectrum"}
		\label{res:zug_spectrum}
	\end{minipage}
\end{figure}

Cледуя техническому описанию, устанавливаем на генераторе режим
подачи периодических импульсов синусоидальной формы. Частоту несущей
устанавливаем $\nu_0 = 50 \text{кГц}$, период повторения $T = 1 \text{мс}$ ($\nu_{повт} = 1 \text{кГц}$), число периодов в одном импульсе $N = 5$ (длительность импульса $\tau = N/\nu_{0} = 100$ мкс).


\begin{figure}[H]
	\centering
	\begin{minipage}[b]{.33\textwidth}
		\centering
		\includegraphics[width=0.9\linewidth]{"photo/zug1"}
		\label{photo:zug}
		\caption{Изменения спектра при уменьшении $\tau$}
	\end{minipage}%
	\begin{minipage}[b]{.33\textwidth}
		\centering
		\includegraphics[width=0.9\linewidth]{"photo/zug2"}
	\end{minipage}%
	\begin{minipage}[b]{.33\textwidth}
		\centering
		\includegraphics[width=0.9\linewidth]{"photo/zug3"}
	\end{minipage}
\end{figure}

К сожалению нам не удалось добиться устойчивой картины на экране осциллографа. Скорее всего это связано с генератором импульсов. Из-за этого следующее изучение спектра становится близким к невозможному.
	
\subsubsection*{ Исследование спектра гармонических сигналов, модулированных по амплитуде}
Теоретическое описание гармонического сигнала, модулированного по амплитуде приведено на рисунке \ref{res:am}.

\begin{figure}[H]
	\includegraphics[width=0.5\linewidth]{"res/am"}
	\caption{Спектр гармонических сигналов, модулированных по амплитуде.}
	\label{res:am}
\end{figure}

Настраиваем генератор на частоту несущей $\nu_{0} = 25 \text{кГц}$, частоту модуляции
$\nu_{\text{мод}} = 1 \text{кГц}$ и глубину модуляции $m = 0.5$. Фотографии экрана электронного осциллографа приведены на рисунке \ref{photo:am}.	

\begin{figure}[H]
	\centering
	\begin{minipage}[b]{.33\textwidth}
		\centering
		\includegraphics[width=0.9\linewidth]{"photo/am1"}
		\label{photo:am}
		\caption{Изменения спектра при увеличении $\nu_{0}$}
	\end{minipage}%
	\begin{minipage}[b]{.33\textwidth}
		\centering
		\includegraphics[width=0.9\linewidth]{"photo/am2"}
	\end{minipage}%
	\begin{minipage}[b]{.33\textwidth}
		\centering
		\includegraphics[width=0.9\linewidth]{"photo/am3"}
	\end{minipage}
\end{figure}

Меняя на генераторе глубину модуляции $m$ в диапазоне от $10\%$ до
$100\%$, измерим отношение $\frac{a_{\text{бок}}}{a_{\text{осн}}}$ амплитуд боковой и основной спектральных линий. Строим график зависимости $\frac{a_{\text{бок}}}{a_{\text{осн}}}(m)$ (см. рис. \ref{fig:v21}).

\noindent \begin{minipage}[B]{.5\textwidth}
	\begin{figure}[H]
		\centering
		\includegraphics[width=1.1\linewidth]{"gen/fig-v21.pdf"}
		\label{fig:v21}
		\caption{График зависимости $\frac{a_{\text{бок}}}{a_{\text{осн}}}(m)$}
	\end{figure}
\end{minipage}%
\begin{minipage}[B]{.5\textwidth}
	\begin{table}[H]
		\footnotesize
		\input{gen/tab-v21.tex}
		\caption{Данные}
		\label{tab:v21}
	\end{table}
\end{minipage}

\begin{table}[H]
	\footnotesize
	\input{gen/tab-v21-mnk.tex}
	\caption{Обработка МНК}
	\label{tab:v21-mnk}
\end{table}

Погрешность эксперимента коррелирует с погрешностью коэффициента наклона. Следовательно, рассчитаем количественный критерий точности:

$$\mathcal{C} \approx \frac{\Delta a}{a} \approx 3 \% $$

\subsection*{Вывод}

В работе было проведено изучение спектрального состава различных периодических электрических сигналов. Примечательно, что эксперимент получается с очень хорошей точностью (относительная погрешность порядка нескольких процентов). 

К сожалению во второй части работы нам так и не удалось добиться устойчивой картины на экране осциллографа. Скорее всего это связано с генератором импульсов.
	
\end{document}