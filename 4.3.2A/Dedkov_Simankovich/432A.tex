\documentclass{beamer}

\usefonttheme{serif}

\setbeamertemplate{footline}[frame number]{}
\setbeamertemplate{navigation symbols}{}

\usecolortheme{default}
%\setbeamercolor{block title}{bg=lily,fg=black}
%\setbeamercolor{block body}{bg = blue!10, fg = black}
\setbeamertemplate{itemize item}[square]
\setbeamercolor{itemize item}{fg = cyan}
\setbeamercolor{enumerate item}{fg = cyan}

\usetheme{default}

%\setbeamercolor{titlelike}{fg=cyan}
%Information to be included in the title page:
\title{Sample title}
\author{Anonymous}
\institute{Overleaf}
\date{2021}

\title[About Beamer] %optional
{Diffraction of light by ultrasonic waves}

%\subtitle{A short story}

\author[Arthur, Doe] % (optional, for multiple authors)
{A.~Simankovich \and D.~Dedkov }

\institute[VFU] % (optional)
{
	Moscow Institute of Physics and Technology
}

\date[VLC 2023] % (optional)
%{Very Large Conference, April 2021}

%\logo{\includegraphics[height=1cm]{overleaf-logo}}

\begin{document}
	
	\frame{\titlepage}
	
	\begin{frame}
		\frametitle{Abstract}
		Acoustic waves in liquids cause density changes with spacing determined by the
		frequency and the speed of the sound wave.
		
			
		TODO TODO
		To determine the speed of sound in various liquids at room temperature
		To determine the compressibility of the given liquids

	\end{frame}
	
		\begin{frame}[plain,c]
		
		\begin{center}
			\huge \usebeamercolor[fg]{frametitle} Introduction
		\end{center}
		
	\end{frame}
		
	
	\begin{frame}
		\frametitle{Diffraction on periodic structures}
		
		TODO: CENTERING
		
		\includegraphics[width=4cm]{res/periodic.png}
		\includegraphics[width=4cm]{res/diffraction_general.png}
		
		
		General theory shows that when a plane light wave is normally incident on the grating, the diffracted light has maximum at diffraction angles:

		$$d \sin{\psi_m} = m \lambda,\; (m = 1, 2, ...)$$
		
		$<WAT>$
		However, especially the methods of obtaining such structures that are of practical interest.

		$<\backslash WAT>$
	\end{frame}


	\begin{frame}
		\frametitle{Diffraction by ultrasonic waves}
		
		TODO: CENTERING
		
		\includegraphics[width=4cm]{res/acoustic_grating.png}
		\includegraphics[width=4cm]{res/n_x.png}
		
		One example of such system is \textbf{ultrasonic waves grating}. 
		Acoustic waves in liquids cause density changes with spacing determined by the
		frequency and the speed of the sound wave.
		Local changes in the water density lead to a change in the refractive index $n \approx n_0 (1 + \cos{\frac{2\pi}{\Lambda}x})$, where $\Lambda$ - ultrasonic wavelength.
		
		This forms a \textbf{phase diffraction grating}:
		
		\begin{equation}
			\varphi(x) = knL = \varphi_0 (1 + a \cos{\frac{2\pi}{\Lambda}x}).
			\label{eq:diffraction}
		\end{equation}
	\end{frame}

	
	\begin{frame}
		\frametitle{Velocity of the ultrasonic wave}
		
		If $\nu$ is the frequency of the $<WAT>$ crystal $<\backslash WAT>$, the velocity $v$ of ultrasonic wave in the
		liquid is given by:
		\begin{equation}
			v = \nu \Lambda.
			\label{eq:vel}
		\end{equation}
		
		Thus, by measuring the angle of diffraction $\theta_n$, the order of diffraction $n$, the
		light wavelength, the length of ultrasonic wave in the liquid can be determined
		and then, knowing the frequency of sound wave, its velocity $v$ can be obtained. 
	\end{frame}

	\begin{frame}
		\frametitle{Phase grating spatial structure observation}
		
				Observation of the spatial structure of the phase grating is complicated. The central problem is that phase grating intensity is \textbf{constant}. Indeed, complex transmittance has the following form:
				$$t(x) = e^{i\varphi(x)},$$ with intensity
				$$I(x) = |f_0(x)|^2 = 1.$$
				
				$<WAT>$By the way$<\backslash WAT>$, there are some methods that allow you to observe the spatial structure.

	\end{frame}

	\begin{frame}
		\frametitle{Dark-field method}
		
		One of the most popular methods is called \textbf{dark-field method}. The idea behind is filtering the central maximum with using the screen, we get:
		
		\begin{equation*}
				f_0(x) = e^{i m \cos{\Omega x}} \approx 1 +  i m \cos{\Omega x} \overset{\textbf{filtering}}{=}  i m \cos{\Omega x}.
		\end{equation*}
		Then the intensity of the filtered light is the following:
		
		\begin{equation*}
				I_f(x) = m^2 \cos^2{\Omega x} = \frac{m^2}{2} (1 + \cos{2 \Omega x}) \neq 1.
				\label{eq:dark_field}
		\end{equation*}
		$\cos{2 \Omega x}$ implies that distance between interference lines is half of grating period.
	
		Another methods, such as \textbf{phase constrast method}, is based on the transition from a phase grating to an amplitude one.
	\end{frame}

	\begin{frame}
		\frametitle{Phase and Amplitude Grating with Uniform Beam}
		
		\begin{figure}
			\centering
			\includegraphics[width=0.5\linewidth]{res/amplitude_phase_grating}
			\caption{Phase (a) and amplitude (b) grating vector diagrams}
			\label{fig:amplitudephasegrating}
		\end{figure}
		
		The grating modulation functions are respectively ($i$ - matters):
		$$	t(x) = e^{i m \cos \Omega x} \approx 1 + \frac{im}{2}e^{i \Omega x} + \frac{im}{2}e^{-i \Omega x},$$
		
		$$	t(x) = 1 + m \cos{\Omega x} = 1 + \frac{m}{2}e^{i \Omega x} + \frac{m}{2}e^{-i \Omega x}. $$
	
	\end{frame}	

		\begin{frame}
		\frametitle{Shifting the screen}

		In other words, we can see interaction of three waves with the following modulations:
		 $1, \frac{im}{2}e^{i \Omega x},\frac{im}{2}e^{-i \Omega x}$. This gives the following waves equations: 
		$$e^{i kz}, \frac{im}{2}e^{i k(z\cos{\psi} + x\sin{\psi})},\frac{im}{2}e^{i k(z\cos{\psi} - x\sin{\psi})} \; \left(\sin{\psi} = \pm\frac{\Omega}{k}  \right)$$
		
		
		It follows from the above that there is a possibility of the phase-amplitude grating transition. Indeed, let the screen plane shift by $\Delta z$, this will result in the phase shift:
		
		$$\text{Phase shift} = k\Delta z(1 - \cos{\psi}).$$
		
		
		If $\Delta z = \frac{\pi}{2} + 2\pi n$, then the central wave $E_0$ has been rotated by $\frac{\pi}{2}$ and phase-amplitude transition has been occurred.
	\end{frame}

	\begin{frame}[plain,c]
		
		\begin{center}
			\huge \usebeamercolor[fg]{frametitle} Measurements and Results
		\end{center}
		
	\end{frame}

	\begin{frame}
		\frametitle{Experimental Setup}
		\begin{figure}
			\centering
			\includegraphics[width=0.8\linewidth]{res/real_setup.png}
		\end{figure}
		
		\begin{itemize}
			\item $f = (30 \pm 0.1)$ cm -- focal length of lenses
			\item $\lambda = (640 \pm 20)$ nm -- light wave length (red)
			\item $\nu \in [1.0; 5.0]$ MHz -- ultrasonic generator frequency range
			
		\end{itemize}		
	\end{frame}
	

	\begin{frame}
		\frametitle{Experimental Setup}
		\begin{figure}
			\centering
			\includegraphics[width=10cm]{res/setup1.png}
		\end{figure}
		
		\begin{columns}
			\column{0.35\textwidth}
			\begin{enumerate}
				\item[$\bullet$] $L$ - light source
				\item[$\bullet$] $\Phi$ - light filter
				\item[$\bullet$] $K$ - condenser
				\item[$\bullet$] $O_1, O_2$ - lenses
				\item[$\bullet$] $Q$ - cuvette
				\item[$\bullet$] $B$ - measurement screw
				\item[$\bullet$] $M$ - microscope
			\end{enumerate}
			\column{0.70\linewidth}
			Light source $L$ illuminates slit $S$ through $\Phi$ and $K$. Parallel light beam passes through cuvette $Q$. Ultrasonic signal from generator is supplied to quartz piezoelectric plate. Light interacts with ultrasonic wave, resulting in interference pattern in focal plane of $O_2$. Pattern can be observed in microscope.
		\end{columns}
	\end{frame}
	
	\begin{frame}
		\frametitle{Observing diffraction lines}

		\begin{figure}
			\includegraphics[height=4cm]{data/part1/1.jpg}
			\includegraphics[height=4cm]{data/part1/2.jpg}
			\caption{Diffraction lines in microscope}
		\end{figure}
	
		\begin{columns}
			\column{0.6\textwidth}
			As described above, we can see interference pattern on different frequencies. We use glass with thin notches and measurement screw $B$ to determine distances between diffraction lines.
			\column{0.4\textwidth}
			\begin{figure}
				\centering
				\includegraphics[width=0.5\linewidth]{res/measurement_glass.png}
				\vspace{-5pt}
				\caption{Notches scheme}
			\end{figure}
		\end{columns}
	
	\end{frame}


	\begin{frame}
		\frametitle{Cuvette length variation}

		Using screw on cuvette we can change it's length. We setup generator frequency to achive well-recognizable interference pattern ($\nu = 1.16$ MHz). In that case wave becomes standing. Taking into account that changing cuvette length to $\Lambda / 2$ again gives us standing wave, we estimate: 
		
		$$ \Lambda / 2 = 700 \; \text{mkm} \Rightarrow v = \Lambda \nu = 1600 \;\text{m/s}.$$
		
	\end{frame}


	\begin{frame}
		\frametitle{Diffraction maxima}
		\begin{columns}
			\column{0.6\textwidth}
			\begin{figure}
				\centering
				\includegraphics[width=1.1\linewidth]{gen/part1_xn.pdf}
				\caption{Maxima positions for different frequencies}
				\label{fig:part1_xn}
			\end{figure}
			\column{0.4\textwidth}
			Slope coefficient $k = \frac{dx}{dm}$.
			
			From (\ref{eq:diffraction}) and (\ref{eq:vel}):
			$$ v = \frac{f \lambda \nu m}{x_m} = \frac{f \lambda \nu}{k}.$$
			
			After evaluating $k$ for every frequency and taking average:
			$$ v = (1480 \pm 30)\; \text{m/s}.$$
			
		\end{columns}
				
	\end{frame}

	\begin{frame}
		\frametitle{Experimental Setup}
		\begin{figure}
			\centering
			\includegraphics[width=10cm]{res/setup2.png}
		\end{figure}
		
		\begin{columns}
			\column{0.35\textwidth}
			\begin{enumerate}
				\item[$\bullet$] $S$ - slit
				\item[$\bullet$] $O_1, O_2$ - lenses
				\item[$\bullet$] $Q$ - cuvette
				\item[$\bullet$] $B$ - measurement screw
				\item[$\bullet$] $O$ - auxiliary lens
				\item[$\bullet$] $P$ - clear-image plane
				\item[$\bullet$] $M$ - microscope
			\end{enumerate}
			\column{0.70\linewidth}
			Comparing with previous setup, we add auxiliary lens $O$. It creates clear image of objects in cuvette $Q$ in plane $P$. We use wire to cut off central maximum in $O_2$ focal plane. Interference pattern can be observed in microscope.
		\end{columns}
	\end{frame}
	
	\begin{frame}
		\frametitle{Dark-field calibration}
		
		\begin{figure}
			\includegraphics[width=4cm]{data/part2/ruler.jpg}
			\caption{Calibration ruler in microscope}
		\end{figure}
	
		To determine distance between interference lines we calibrate microscope scale using glass with millimeter notches. Scale coefficient $\gamma = 0.6$ mm/div.
		
	\end{frame}
	

	\begin{frame}
		\frametitle{Observing dark-field lines}
		
		\begin{figure}
			\includegraphics[height=3.5cm]{data/part2/1.jpg}
			\includegraphics[height=3.5cm]{data/part2/2.jpg}
			\includegraphics[height=3.5cm]{data/part2/3.jpg}
			\caption{Dark-field maxima in microscope}
		\end{figure}
		
		Picture is continuous before central maximum is cut off. When it's blocked we can observe interference pattern.
		
		Interference lines on pictures are indexed. Microscope scale is marked in blue.
		
	\end{frame}
	
	
	\begin{frame}
		\frametitle{Dark-field maxima}
		
		\begin{columns}
			\column{0.65\linewidth}
			\begin{figure}
				\includegraphics[width=1.1\linewidth]{gen/part2_xn.pdf}
				\caption{Maxima positions for different frequencies}
			\end{figure}
			\column{0.35\linewidth}
			From ($\ref{eq:dark_field}$) we obtain
			$$\Lambda = 2 \frac{x_m}{m} = 2 k.$$
			Therefore:
			$$ v = 2 k \nu.$$
			
			
			Evaluating $k$ for all frequencies and taking average:
			$$ v = (1574 \;\pm \; 12) \; \text{m/s}. $$
			
		\end{columns}
	
		
	\end{frame}
	
	

	\begin{frame}
		\frametitle{Acknowlegements}
	\end{frame}
	
	
\end{document}