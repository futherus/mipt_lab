\section*{Введение}

Вещество может находиться в трёх агрегатных состояниях - твёрдом, жидком и газообразном, причём эти состояния последовательно сменяются по мере возрастания температуры. Если и дальше нагревать газ, то сначала молекулы диссоциируют на атомы, а затем и атомы распадаются на электроны и ионы, так что газ становится ионизованным, представляя собой смесь из свободных электронов и ионов, а также нейтральных частиц. Если степень ионизации газа (отношение числа ионизованных атомов к их полному числу) оказывается достаточно велика, то такой газ может обладать качественно новыми свойствами. Поведение заряженных частиц приобретает коллективный характер, так что описание свойств среды не может быть сведено к описанию обычного газа, содержащего некоторое количество отдельных заряженных частиц. Такое состояние ионизованного газа называется \textit{плазмой}.