\section*{Введение}

Из опытов известно, что вещество может реагировать на внешнее магнитное поле. Из-за внутренней структуры парамагнетики (<<усиливают>> действие внешнего магнитного поля) и диамагнетики (стараются <<скомпенсировать>> внешнее магнитное поле) довольно слабо реагируют с полем. Ферромагнетики, наоборот, проявляют сильное взаимодействие.

Согласно уравнениям Максвелла, магнитное поле создаётся движущимися зарядами. Магнитное поле также создаётся орбитальным движением электронов вокруг атомов и собственным вращением электронов (спином) и ядер. Полноценное описание магнитных свойств вещества возможно только при применении квантовой механики. В данной работе приводится полуклассическая модель магнитного поля в веществе. Предполагается, что электроны и ядра не обладают собственным вращением.

