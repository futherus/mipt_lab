\section*{Введение}

Из опытов известно, что вещество может реагировать на внешнее магнитное поле. Из-за внутренней структуры парамагнетики <<усиливают>> действие внешнего магнитного поля, а диамагнетики стараются его <<скомпенсировать>>. Для диа- и парамагнетиков среднее изменение магнитного поля составляет $10^{-8} \div 10^{-4}$ от внешнего. Ферромагнетики, напротив, проявляют сильное взаимодействие и усиливают внешнее магнитное поле в $10^3 \div 10^4$ раз.

Согласно уравнениям Максвелла, магнитное поле создаётся движущимися зарядами. Магнитное поле также создаётся орбитальным движением электронов вокруг атомов и собственным вращением электронов (спином) и ядер. Полноценное описание магнитных свойств вещества возможно только при применении квантовой механики. В данной работе приводится полуклассическая модель магнитного поля в веществе. Предполагается, что электроны и ядра не обладают собственным вращением.

