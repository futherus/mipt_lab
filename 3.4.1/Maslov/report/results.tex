\section*{Выводы}

В работе была измерена магнитная восприимчивость: \\
Медь $\chi = -(1.08 \pm 0.03) \cdot 10^{-5}$. \\
Алюминий $\chi = +(2.17 \pm 0.03) \cdot 10^{-5}$. \\
Графит $\chi = +(21.5 \pm 1.1) \cdot 10^{-5}$. \\
Вольфрам $\chi = +(7.24 \pm 0.26) \cdot 10^{-5}$.

Табличные значения: \\
Медь $\chi = -1.03 \cdot 10^{-5}$. \\
Алюминий $\chi = +2.3 \cdot 10^{-5}$. \\
Графит $\chi = +10.5 \cdot 10^{-5}$. \\
Вольфрам $\chi = +17.6 \cdot 10^{-5}$.

Достаточно точно получилось измерить магнитную восприимчивость для меди и алюминия. Расхождения результатов для графита и вольфрама возможны из-за наличия в веществах примесей, например ферромагнетиков, которые могут значительно повлиять на результат измерения. Также на результат могло повлиять несовершенство используемой модели, которая даёт хорошее качественное описание, но количественное совпадает с более точными моделями только по порядку величины, что и наблюдается для графита и вольфрама.