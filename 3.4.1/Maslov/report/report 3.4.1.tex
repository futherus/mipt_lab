\documentclass[12pt,a4paper]{extreport}
\usepackage[a4paper, top=1.5cm, bottom=1.5cm, left=1.5cm, right=1.5cm]{geometry}
\setlength{\parindent}{1.25cm} % Отступ красной строки
\usepackage{listings} 
\usepackage{caption}

\usepackage{hyperref}
\usepackage{booktabs}
\usepackage{lipsum}

\usepackage{graphicx} % Для того, чтобы вставлять картинки.
\usepackage{wrapfig} % Картинка, обтекаемая текстом.
\usepackage{amssymb}
\usepackage{amsmath} % Чтобы вставлять обычный текст в формулу с помощью \text. Фигурная скобка в системе уравнений.

\usepackage{multicol} % Для написания текста в несколько колонок

\usepackage{setspace} % Для изменения межстрочного интервала внутри текста. \begin{spacing}{0.8}.
	
\usepackage{float}   % Чтобы была опция таблиц H, запрещающая им бегать по документу
\restylefloat{table}

\usepackage{gensymb} % Геометрические символы. (градусы \degree)..

\usepackage[warn]{mathtext} % Русские символы в формулах. Нужно писать до пакета babel. Указывает, что в формулах используются символы кириллицы, которые по умолчанию печатаются прямым шрифтом.

\usepackage[T2A]{fontenc} % Установить кодировку шрифта для отображения кириллицы в формулах.
\usepackage[utf8]{inputenc}

\usepackage[russian]{babel} % Для переноса текста. Нельзя указывать одновременно russian и english, так как использует язык, который стоит правее.

\usepackage{indentfirst} % Красная строка в первом абзаце.

\usepackage{comment} % Для многострочный комментариев.

%\DeclareSymbolFont{T2Aletters}{T2A}{cmr}{m}{it} % Сделать так, чтобы кириллица в формулах печаталась курсивом

\linespread{1.25} % Межстрочный интервал. По умолчанию 1.0

% Объявляем новую команду для переноса строки внутри ячейки таблицы
\newcommand{\specialcell}[2][c]{%
\begin{tabular}[#1]{@{}c@{}}#2\end{tabular}}

\newcommand{\figref}[1]{(см. рис. \ref{#1})}
\newcommand{\e}[1]{\text{$\cdot10^{#1}$}}

\begin{document}

	\begin{center}
		\large
		\textsc{Лабораторная работа №3.4.1}
		
		\LARGE
		\textbf{\textsc{Диа- и парамагнетики}}
		\\[5mm]
		
		\large
		Симанкович Александр\\
		Маслов Артём\\
		Б01-104
		\\[3mm]
		29.09.2022
	\end{center}
	
	\section*{Аннотация}

В работе приводится эмпирическая теория спектрального анализа сигналов. Исследуется спектр периодической последовательности прямоугольных импульсов и синусоидальных цугов. Исследуется спектр амплитудно-модулированного сигнала и сигнала, модулированного по фазе.
	
	\section*{Введение}

Вещество может находиться в трёх агрегатных состояниях - твёрдом, жидком и газообразном, причём эти состояния последовательно сменяются по мере возрастания температуры. Если и дальше нагревать газ, то сначала молекулы диссоциируют на атомы, а затем и атомы распадаются на электроны и ионы, так что газ становится ионизованным, представляя собой смесь из свободных электронов и ионов, а также нейтральных частиц. Если степень ионизации газа (отношение числа ионизованных атомов к их полному числу) оказывается достаточно велика, то такой газ может обладать качественно новыми свойствами. Поведение заряженных частиц приобретает коллективный характер, так что описание свойств среды не может быть сведено к описанию обычного газа, содержащего некоторое количество отдельных заряженных частиц. Такое состояние ионизованного газа называется \textit{плазмой}.
	
	\section*{Теория}

\subsection*{Уравнение колебательного контура}

При рассмотрении физических процессов в электрических цепях используются следующие предположения. Во-первых, все элементы электрической цепи считаются \textit{идеальными}. Предполагается, что у катушек индуктивности и конденсаторов нет омического сопротивления, источник напряжения обладает нулевым сопротивлением, а источник тока бесконечно большим, и т.д. Такое представление упрощает анализ физических процессов в электрических цепях. Если же такие предположения вносят большую погрешность, то в схему добавляются дополнительные идеальные элементы, которые учитывают особенности физических процессов в конкретных случаях. 

Во-вторых, рассматриваются \textit{квазистационарные процессы}. Известно, что электромагнитные колебания распространяются с конечной скоростью. В данной работе рассматриваются такие электрические цепи, в которых время установления электромагнитных колебаний пренебрежимо мало. 

\begin{wrapfigure}{left}{0.22\textwidth}
	\vspace{-10pt}
	\centering
	\includegraphics[width=0.2\textwidth]{../res/seq_osc_circ.png}
	\caption{Последовательный колебательный контур}
	\label{fig:seq_osc_circ}
\end{wrapfigure}

Рассмотрим последовательный колебательный контур без источника ЭДС (рис. $\ref{fig:seq_osc_circ}$).  Пусть напряжение на конденсаторе меняется по закону $U = U(t)$. Тогда, согласно второму правилу Кирхгофа, сумма падений напряжений равна 0:
$$
L \frac{dI}{dt} + U + RI = 0
$$
Ток через конденсатор определяется из соотношения
$$
I = \frac{dq}{dt} = C\frac{dU}{dt}
$$
Тогда получим дифференциальное уравнения второго порядка, описывающее \textit{свободные колебания} в линейной системе:
$$
LC \frac{d^2 U}{dt^2} + RC \frac{dU}{dt} + U = 0
$$
Данное уравнение можно переписать в виде:
\begin{equation*}
	\ddot{U} + 2\gamma \dot{U} + \omega_0^2 U = 0
	\label{eq:diff_eq}
\end{equation*}
где введены обозначения $\gamma = \frac{R}{2L}$ -- \textit{коэффициент затухания}, $\omega_0 = \frac{2\pi}{T_0} = \frac{1}{\sqrt{LC}}$ -- \textit{собственная частота} колебательной системы, $T_0 = 2\pi \sqrt{LC}$ -- \textit{период собственных колебаний}.

Найдём решение однородного дифференциального уравнения с постоянными коэффициентами. Запишем характеристическое уравнение:
$$
\lambda^2 + 2\gamma \lambda + \omega_0^2 = 0
$$
$$
D_1 = \frac{D}{4} = \gamma^2 - \omega_0^2
$$
В зависимости от знака дискриминанта квадратного уравнения возможны три случая.

\begin{enumerate}
	\item \textit{Затухающие колебания}.
	
	Рассмотрим случай, когда $D_1 < 0$. Тогда $0 < \gamma < \omega_0$, что эквивалентно
	$$
	0 < R < 2 \sqrt{\frac{L}{C}} = R_{кр}
	$$
	Сопротивление $R_{кр} = 2 \sqrt{\frac{L}{C}}$ называется критическим, а $\rho = \sqrt{\frac{L}{C}}$ -- волновым.
	
	В рассматриваемом случае характеристическое уравнение имеет два комплексных корня 
	$$
	\lambda_{1,2} = -\gamma \pm j\sqrt{\omega_0^2 - \gamma^2}
	$$
	Величину $\omega = \sqrt{\omega_0^2 - \gamma^2}$ называют частотой свободных колебаний. Решением уравнения будет
	$$
	U(t) = U_1 \cdot e^{-\gamma t} \cdot e^{-j\omega t} + U_2 \cdot e^{-\gamma t} \cdot e^{j\omega t}
	$$
	где $U_1$ и $U_2$ -- произвольные постоянные.
	
	Полученное уравнение можно представить в виде
	\begin{equation*}
		U(t) = U_0 e^{-\gamma t} sin(\omega t + \varphi_0)
		\label{eq:damp_osc}
	\end{equation*}
	Данное уравнение является гармоническим с фазой $\omega t + \varphi_0$ и экспоненциально убывающей амплитудой $U_0 e^{-\gamma t}$.
	
	График зависимости напряжения от времени представлен на рисунке $\ref{fig:damp_osc}$.
	
	\begin{figure}[H]
		\vspace{-10pt}
		\centering
		\includegraphics[width=0.3\textwidth]{../res/damp_osc.png}
		\caption{Затухающие колебания}
		\label{fig:damp_osc}
	\end{figure}

	С точки зрения математики данный колебательный процесс не периодичен. Тем не менее функция $U(t)$ обращается в ноль или достигает экстремумов через один и тот же промежуток времени, который называю \textit{периодом затухающих колебаний}.
	
	\item \textit{Критический режим}. 
	
	Рассмотрим случай, когда $D_1 = 0$. Тогда 
	$$
	\gamma = \omega_0
	$$
	Характеристическое уравнение имеет один корень
	$$
	\lambda = - \gamma
	$$
	Решением исходного уравнения будет
	$$
	U(t) = U_0 e^{-\gamma t}
	$$
	где $U_0$ -- постоянная, определяемая из начальных условий.
	
	Заметим, что данный режим физически не реализуем, так как равенство $\gamma = \omega_0$ не может быть выполнено точно. Данный случай нужно рассматривать как переходный между затухающими колебаниями и апериодическим режимом.
	
	\item \textit{Апериодический режим}. 
	
	Рассмотрим случай, когда $D_1 > 0$. Тогда $0 < \omega_0 < \gamma$. Характеристическое уравнение имеет два действительных корня
	$$
	\lambda_{1,2} = -\gamma \pm \sqrt{\omega_0^2 - \gamma^2}
	$$
	Решением дифференциального уравнения будет
	$$
	U(t) = e^{-\gamma t} \cdot (U_1 e^{-j\omega t} + U_2 e^{j\omega t})
	$$
	где $U_1$ и $U_2$ -- произвольные постоянные.
\end{enumerate}

\subsection*{Характеристики затухающих колебаний}

Важными характеристиками колебательных систем являются добротность $Q$ и логарифмический декремент $d$.

Логарифм отношения амплитуд колебаний в двух последовательных максимумах называется логарифмическим декрементом
$$
d = \ln{\left(\frac{A_n}{A_{n+1}}\right)}
$$
Определив положения последовательных максимумов из формулы $\ref{eq:damp_osc}$, можно получить следующее соотношение
\begin{equation*}
	d = \gamma T
	\label{eq:log_dec}
\end{equation*}
где $T$ -- период затухающих колебаний.

\textit{Постоянной времени затухания} $\tau$ называется время, за которое амплитуда колебаний убывает в $e$ раз. Коэффициент затухания и постоянная времени связаны соотношением
\begin{equation*}
	\tau = \frac{1}{\gamma}
	\label{eq:const_time}
\end{equation*}

Из уравнений $\ref{eq:log_dec}$ и $\ref{eq:const_time}$ следует, что логарифмический декремент можно определить как число полных колебаний $N = \frac{\tau}{T}$ за время затухания $\tau$:
$$
d = \frac{1}{N}
$$

Добротностью колебательной системы $Q$ называется
$$
Q \equiv \frac{\pi}{d} = \frac{\pi}{\gamma T} = \frac{\omega}{2 \gamma}
$$ 
Чем выше добротность колебательной системы, тем меньше будут потери энергии. Докажем данное утверждение.

Амплитуда колебаний напряжение за период уменьшается в $e^{\gamma T}$ раз. Полная энергия системы $W$ определяется как максимальная энергия электрического поля конденсатора или магнитного поля индуктивности
$$
W = \frac{CU^2}{2} = \frac{LI^2}{2}
$$
Из этого соотношения видно, что за период энергия системы уменьшается как квадрат амплитуды в $e^{2\gamma T}$ раз.
Тогда потери энергии системы равно
$$
\Delta W = W(t_0) - W(t_0 + T) = (1 - e^{-2\gamma T}) W(t_0)
$$
Если затухание мало, то есть $\gamma T \ll 1 \Rightarrow Q \gg 1$, то экспоненту можно разложить по формуле Тейлора
$$
\Delta W \approx 2 \gamma T W
$$
$$
\frac{W}{\Delta W} = \frac{1}{2\gamma T} = \frac{1}{2\pi}Q
$$
Таким образом, добротность с энергетической точки зрения определяет отношении энергии системы к потерям за период.

\subsection*{Вынужденные колебания}

Если в цепь последовательного колебательного контура включен гармонический источник ЭДС $\varepsilon(t) = \varepsilon_0 \cos{(\omega t)}$, то
$$
\ddot{U} + 2\gamma \dot{U} + \omega_0^2 U = \frac{\varepsilon_0}{LC} \cos{(\omega t)}
$$
Решением неоднородного дифференциального уравнения будет сумма однородного и частного решений
$$
U(t)_{общ} = U(t)_{одн} + U(t)_{част}
$$
%TODO: однородное + частное
Однородным решением будут затухающие колебания
$$
U(t)_{одн} = U_0 e^{-\gamma t} sin(\omega t + \varphi_0)
$$
Найдем частное решение неоднородного уравнения.

Поэтому, через большой промежуток времени, напряжение будет изменяться по закону $U(t)_{}$

\subsection*{Метод комплексных амплитуд}

\subsection*{Метод векторных диаграмм}

\subsection*{Мощность колебательной системы}

\subsection*{Резонанс в колебательном контуре}

%TODO: ФЧХ, АЧХ

	
	\section*{Схема экспериментальной установки}

Схема экспериментальной установки изображена на рисунке:

\begin{figure}[H]
	\centering
	\includegraphics[width=0.7\textwidth]{../res/exp scheme.png}
\end{figure}

Переменное магнитное поле создаётся соленоидом, подключенным к генератору звуковой частоты ЗГ. Соленоид намотан на полый цилиндрический каркас из поливинилхлорида 1, в который вставлен медный цилиндр 2. Для измерения магнитного поля внутри цилиндра используется измерительная катушка 3. Действующие значения тока и напряжения в цепи измеряются соответственно амперметром $A$ и вольтметром $V$. Для измерения сдвига фаз используется осциллограф ЭО. Один вход осциллографа подключен к резистору, напряжение на котором пропорционально току в цепи. Второй вход подключен к измерительной катушке.

ЭДС индукции, возникающая в измерительной катушке равна
$$
U = - SN \frac{dB_1(t)}{dt} = -i \omega \mu_0 SN H_1 e^{i\omega t}
$$
Действующее значение напряжения, измеряемое вольтметром
$$
U = \frac{SN \omega}{\sqrt{2}} \mu_0 |H_1|
$$
Магнитное поле внутри цилиндра пропорционально напряжению и обратно пропорционально частоте:
$$
|H_1| \propto \frac{U}{\nu}
$$
Магнитное поле снаружи катушки пропорционально пропускаемому току:
$$
|H_0| \propto I
$$
Тогда 
\begin{equation}
	\frac{|H_1|}{|H_0|} = C \cdot \frac{U}{\nu I}
	\label{eq:UInu}
\end{equation}
где $C$ -- некоторая константа. Эта константа может быть экспериментально измерена при малых частотах, так как при $\nu \rightarrow 0$, выполняется $\frac{|H_1|}{|H_0|} \rightarrow 1$.

Так как материал, из которого изготовлен цилиндр может содержать примеси, то в работе измеряется проводимость данного цилиндра $\sigma$. Для измерения $\sigma$ измеряется зависимость сдвига фаз между $H_1$ и $H_0$ от $\nu$ в областях низких, высоких частот. Также измеряется индуктивность экрана $L$ от частоты $\nu$.

Важно отметить, что, поскольку сигнал на измерительной катушке 3 пропорционален не полю, а его производной, то сдвиг фаз $\psi = \varphi - \pi/2$, где $\varphi$ -- измеренная на экране осциллографа разность фаз между входами $I$ и $II$.
	
	\section*{Методика измерений}

Сначала проводится градуировка электромагнита. Измеряется зависимость магнитной индукции от силы тока $B(I)$. Измерения производятся милливеберметром и миллитесламетром.

После этого для каждого образца измеряют зависимость действующей на образец силы от тока в электромагните $F(I)$. Измерения производятся при увеличении тока от 0 до максимального значения, а затем при уменьшении тока.
	
	\section*{Оборудование}

\begin{enumerate}
	\item Цифровой генератор сигналов АКИП-3409/4.
	
	\item Цифровой осциллограф SIGLENT АКИП 4131/1.
\end{enumerate}
	
	\section*{Экспериментальные результаты}

\subsection*{Cпектр периодической последовательности прямоугольных импульсов}
Теоретическое описание спектра периодической последовательности прямоугольных импульсов приведено на рисунке \ref{res:impulse}.

\begin{figure}[H]
	\centering
	\begin{minipage}[b]{.49\textwidth}
		\centering
		\includegraphics[width=0.9\linewidth]{"../res/impulse"}
	\end{minipage}%
	\begin{minipage}[b]{.49\textwidth}
		\centering
		\includegraphics[width=0.9\linewidth]{"../res/impulse_spectrum"}
	\end{minipage}
	\caption{Периодическая последовательность импульсов и её спектр.}
	\label{res:impulse}
	\vspace*{-10pt}
\end{figure}

Настраиваем генератор на прямоугольные импульсы с частотой повторения $\nu_T = 1 \; \text{кГц}$ (период $T = 1 \; \text{мс}$) и длительностью импульса $\tau = \frac{T}{20} = 50 \; \text{мкс}$. Снимки экрана электронного осциллографа приведены на рисунках \ref{photo:impulse_nu} и \ref{photo:impulse_tau}. На первом снимке видно, что половина ширины спектра составляет $\Delta \nu = 20$ кГц, что совпадает с рассчитанной выше.

Из соотношения неопределенности $ \Delta \nu \cdot \tau = 1 \Rightarrow \Delta \nu = 20 \; \text{кГц}$, где $\Delta \nu$ -- половина ширины главного спектра.

На снимках \ref{photo:impulse_nu} приведены спектры для различных частот $\nu_T$. Видно, что ширина спектра для них практически не меняется, тогда как меняется расстояние между соседними компонентами спектра.

\begin{figure}[H]
	\centering
	\begin{minipage}[b]{.5\textwidth}
		\vspace*{-10pt}
		\centering
		\includegraphics[width=0.9\linewidth]{"../photos/impulse1.png"}
		\vspace*{-5pt}
		\caption*{$\nu_T= 1$ кГц}
		\vspace*{-10pt}
	\end{minipage}%
	\begin{minipage}[b]{.5\textwidth}
		\vspace*{-10pt}
		\centering
		\includegraphics[width=0.9\linewidth]{"../photos/impulse2.png"}
		\vspace*{-5pt}
		\caption*{$\nu_T= 2$ кГц}
		\vspace*{-10pt}
	\end{minipage}
\end{figure}
\begin{figure}[H]
	\centering
	\begin{minipage}[b]{.5\textwidth}
		\vspace*{-15pt}
		\centering
		\includegraphics[width=0.9\linewidth]{"../photos/impulse3.png"}
		\vspace*{-5pt}
		\caption*{$\nu_T= 3$ кГц}
		\vspace*{-10pt}
	\end{minipage}%
	\begin{minipage}[b]{.5\textwidth}
		\vspace*{-15pt}
		\centering
		\includegraphics[width=0.9\linewidth]{"../photos/impulse4.png"}
		\vspace*{-5pt}
		\caption*{$\nu_T= 4$ кГц}
		\vspace*{-10pt}
	\end{minipage}
	\caption{Изменения спектра при увеличении $\nu$}
	\label{photo:impulse_nu}
\end{figure}

На снимках \ref{photo:impulse_tau} спектры при различных $\tau$. Изменение $\tau$ влияет только на ширину спектра.

\begin{figure}[H]
	\centering
	\begin{minipage}[b]{.5\textwidth}
		\vspace*{-10pt}
		\centering
		\includegraphics[width=0.9\linewidth]{"../photos/impulse1.png"}
		\caption*{$\tau = 50$ мкс}
		\vspace*{-20pt}
	\end{minipage}%
	\begin{minipage}[b]{.5\textwidth}
		\vspace*{-10pt}
		\centering
		\includegraphics[width=0.9\linewidth]{"../photos/impulse5.png"}
		\caption*{$\tau = 100$ мкс}
		\vspace*{-20pt}
	\end{minipage}
\end{figure}

\begin{figure}[H]
	\centering
	\begin{minipage}[b]{.5\textwidth}
		\centering
		\includegraphics[width=0.9\linewidth]{"../photos/impulse6.png"}
		\caption*{$\tau = 150$ мкс}
	\end{minipage}%
	\begin{minipage}[b]{.5\textwidth}
		\centering
		\includegraphics[width=0.9\linewidth]{"../photos/impulse7.png"}
		\caption*{$\tau = 200$ мкс}
	\end{minipage}
	\caption{Изменения спектра при увеличении $\tau$}
	\label{photo:impulse_tau}
\end{figure}

Измерим высоты гармоник спектра при $\nu_T = 1 \; \text{кГц}$, $\tau = 100 \; \text{мкс}$. Теоретические значения высот вычисляются по формуле \eqref{eq:spec_heights}. Снимок спектра приведен на рисунке \ref{photo:harmonics}, график измеренных значений и огибающая приведены на рисунке \ref{fig:harmonics}. Нормировка $a_n$ проведена по высоте пика спектра.

\begin{figure}[H]
\begin{minipage}[H]{.5\textwidth}
		\vspace*{-15pt}
		\centering
		\includegraphics[width=1.0\linewidth]{"../gen/fig-a9.pdf"}
		\caption{Спектр и огибающая}
		\label{fig:harmonics}
\end{minipage}%
\begin{minipage}[H]{.5\textwidth}
		\centering
		\includegraphics[width=1.0\linewidth]{"../photos/impulse5"}
		\caption{Снимок экспериментального спектра}
		\label{photo:harmonics}
\end{minipage}
\end{figure}

Проверим выполнимость соотношения неопределенностей $\Delta \nu \cdot \tau = 1$. Для этого измерим зависимость ширины спектра $\Delta \nu(\tau)$ при постоянной $\nu_T = 1$ кГц.

\begin{figure}
\begin{minipage}[H]{.5\textwidth}
	\begin{figure}[H]
		\centering
		\includegraphics[width=1.2\linewidth]{"../gen/fig-a11.pdf"}
		\label{fig:a11}
	\end{figure}
\end{minipage}%
\begin{minipage}[H]{.5\textwidth}
	\begin{table}[H]
		\centering
		\input{../gen/tab-a11.tex}
	\end{table}
\end{minipage}
\vspace*{-20pt}
\caption{График зависимости $\Delta\nu(\frac{1}{\tau})$ }
\label{tab:a11}
\end{figure}

\begin{table}[H]
	\centering
	\input{../gen/tab-a11-mnk.tex}
	\caption{Обработка МНК}
	\label{tab:a11-mnk}
\end{table}


Погрешность эксперимента коррелирует с погрешностью коэффициента наклона. Следовательно, рассчитаем количественный критерий точности:

$$\mathcal{C} \approx \frac{\Delta a}{a} \approx 0.5 \% $$

\subsection*{Cпектр периодической последовательности синусоидальных цугов}

Теоретическое описание спектра периодической последовательности цугов приведено на рисунке \ref{res:zug}.

\begin{figure}[H]
	\centering
	\begin{minipage}[b]{.49\textwidth}
		\centering
		\includegraphics[width=0.9\linewidth]{"../res/zug"}
	\end{minipage}%
	\begin{minipage}[b]{.49\textwidth}
		\centering
		\includegraphics[width=0.9\linewidth]{"../res/zug_spectrum"}
	\end{minipage}
	\caption{Периодическая последовательность импульсов и её спектр.}
	\label{res:zug}
\end{figure}

На вход осциллографа подаем последовательность цугов. Период повторения $T = 1 \; \text{мс}$ ($\nu_T = 1 \; \text{кГц}$), число периодов в одном импульсе $N = 5$ (длительность импульса $\tau = N/\nu_{0} = 100$ мкс). Несущие частоты возьмем $\nu_0 = \{50, 70, 90\} \; \text{кГц}$. Снимки приведены на рисунке \ref{photo:zug}.

\begin{figure}[H]
	\centering
	\begin{minipage}[b]{.33\textwidth}
		\centering
		\includegraphics[width=0.9\linewidth]{"../photos/zug1"}
	\end{minipage}%
	\begin{minipage}[b]{.33\textwidth}
		\centering
		\includegraphics[width=0.9\linewidth]{"../photos/zug2"}
	\end{minipage}%
	\begin{minipage}[b]{.33\textwidth}
		\centering
		\includegraphics[width=0.9\linewidth]{"../photos/zug3"}
	\end{minipage}
\caption{Изменение спектра при увеличении несущей частоты}
\label{photo:zug}
\end{figure}

На снимках видно, что при увеличении несущей частоты спектр сдвигается, при этом центр спектра находится в $\nu_0$, это характерное отличие последовательности синусоидальных цугов от прямоугольных импульсов. На изменения $T$, $\tau$ спектр реагирует аналогично \ref{photo:impulse_nu}, \ref{photo:impulse_tau}.

Проверим выполнимость соотношения неопределенностей для $T$: $\delta \nu \cdot T = 1$. Для этого измерим зависимость $\delta \nu(T)$.

\begin{figure}[H]
	\begin{minipage}[H]{.5\textwidth}
		\begin{figure}[H]
			\centering
			\includegraphics[width=1.2\linewidth]{"../gen/fig-b16.pdf"}
			\label{fig:b16}
		\end{figure}
	\end{minipage}%
	\begin{minipage}[H]{.5\textwidth}
		\begin{table}[H]
			\centering
			\input{../gen/tab-b16.tex}
		\end{table}
	\end{minipage}
	\vspace*{-20pt}
	\caption{График зависимости $\Delta\nu(\frac{1}{\tau})$ }
	\label{tab:b16}
\end{figure}

\begin{table}[H]
	\centering
	\input{../gen/tab-b16-mnk.tex}
	\caption{Обработка МНК}
	\label{tab:b16-mnk}
\end{table}

Оценим погрешность как
$$\mathcal{C} \approx \frac{\Delta a}{a} \approx 1.0 \%.$$

\subsection*{Спектр гармонических сигналов, модулированных по амплитуде}

Теоретическое описание гармонического сигнала, модулированного по амплитуде приведено на рисунке \ref{res:am}.

\begin{figure}[H]
	\centering
	\begin{minipage}[b]{.49\textwidth}
		\centering
		\includegraphics[width=0.9\linewidth]{"../res/am"}
	\end{minipage}%
	\begin{minipage}[b]{.49\textwidth}
		\centering
		\includegraphics[width=1.0\linewidth]{"../res/am_spectrum"}
	\end{minipage}
	\caption{Амплитудная модуляция и её спектр.}
	\label{res:am}
\end{figure}

Настраиваем генератор на частоту несущей $\nu_{0} = 50 \; \text{кГц}$, частоту модуляции
$\nu_{\text{мод}} = 4 \; \text{кГц}$ и глубину модуляции $m = 0.5$. Частоты модуляции возьмем $\nu_{\text{мод}} = \{4, 8, 12\}$ кГц. Фотографии экрана электронного осциллографа приведены на рисунке \ref{photo:am}.	

\begin{figure}[H]
	\centering
	\begin{minipage}[b]{.33\textwidth}
		\centering
		\includegraphics[width=0.9\linewidth]{"../photos/am1"}
	\end{minipage}%
	\begin{minipage}[b]{.33\textwidth}
		\centering
		\includegraphics[width=0.9\linewidth]{"../photos/am2"}
	\end{minipage}%
	\begin{minipage}[b]{.33\textwidth}
		\centering
		\includegraphics[width=0.9\linewidth]{"../photos/am3"}
	\end{minipage}
	\caption{Изменения спектра при увеличении $\nu_{\text{мод}}$}
	\label{photo:am}
\end{figure}

Меняя на генераторе глубину модуляции $m$ в диапазоне от $10\%$ до
$100\%$, измерим отношение $\frac{a_{\text{бок}}}{a_{\text{осн}}}$ амплитуд боковой и основной спектральных линий. Строим график зависимости $\frac{a_{\text{бок}}}{a_{\text{осн}}}(m)$ (см. рис. \ref{fig:v21}).

\begin{figure}[H]
	\begin{minipage}[B]{.5\textwidth}
			\centering
			\includegraphics[width=1.1\linewidth]{"../gen/fig-v21.pdf"}
	\end{minipage}%
	\begin{minipage}[B]{.5\textwidth}
		\begin{table}[H]
			\centering
			\footnotesize
			\input{"../gen/tab-v21.tex"}
		\end{table}
	\end{minipage}
	\caption{График зависимости $\frac{a_{\text{бок}}}{a_{\text{осн}}}(m)$}
	\label{fig:v21}
\end{figure}


\begin{table}[H]
	\centering
	\footnotesize
	\input{"../gen/tab-v21-mnk.tex"}
	\caption{Обработка МНК}
	\label{tab:v21-mnk}
\end{table}

Погрешность эксперимента коррелирует с погрешностью коэффициента наклона. Следовательно, рассчитаем количественный критерий точности:

$$\mathcal{C} \approx \frac{\Delta a}{a} \approx 3 \% $$


	
	\section*{Выводы}

В работе было подтверждено существование петель гистерезиса, а также измерены их параметры (см \ref{tab:results}, \ref{tab:diff}). Исследованы начальные кривые намагничивания и их свойства.

Дальнейшее исследование может быть направлено на изучение поведения магнетиков в полях с более высокими частотами изменения поля $H$ (в данной работе она составляла $\nu = 50$ Гц). Также возможно провести изучение зависимости свойств от температуры (в данной работе составляла $T \approx 300$ K).

\end{document}