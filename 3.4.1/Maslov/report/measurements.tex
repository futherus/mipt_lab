\section*{Экспериментальные результаты}

Проведем градуировку электромагнита. Измерения проведем милливеберметром M119 и миллитесламетром AKTAKOM ATE-8702. Погрешности данных приборов:
$$ \varsigma_{\text{Вб}} = (0.015 \cdot \Phi + 0.05) \; \text{мВб} \quad \varepsilon_{\text{Тл}} = (0.05 \cdot B + 10) \; \text{мТл}$$
Точность измерения $I_M$ определяется точностью амперметра $A_1$, встроенного в лабораторный блок питания GPR-11H30D: $$\varsigma_{A_1} = 0.005 \cdot I + 0.02 \; \text{А}$$

На рисунке ниже приведена градуировочная кривая.

\begin{figure}[H]
	\includegraphics[]{../gen/electromagnet_BI.pdf}
	\caption{Зависимость поля в зазоре $B$ от протекающего тока $I$}
\end{figure}

Градуировочные графики $B(U)$ совпадают для обоих приборов в пределах погрешности. Зависимость $B(I)$ является не линейной, поэтому последующие измерения будут проводиться при тех же значениях силы тока. Для вычисления значения поля в последующих пунктах используется калибровка с помощью миллитесламетра.

\subsection*{Измерение перегрузок}

С помощью аналитических весов определим значения перегрузки $\Delta P$ при различных значениях поля $B$ в зазоре электромагнита. Измерения будем проводить как в прямом (\textit{рост}), так и в обратном (\textit{падение}) направлении.

На рисунке ниже приведены зависимости перегрузки от квадрата индукции магнитного поля $P(B^2)$ для меди и алюминия.

\begin{figure}[H]
	\includegraphics[]{../gen/cu_al.pdf}
	\caption{Зависимость перегрузки $\Delta P$ от поля в зазоре $B$ для меди и алюминия}
\end{figure}

\begin{table}[h]
	\caption{Параметры графика $\Delta P(B^2)$ для меди}
	\input{../gen/cu_mnk.tex}
\end{table}

\begin{table}[h]
	\caption{Параметры графика $\Delta P(B^2)$ для алюминия}
	\input{../gen/al_mnk.tex}
\end{table}

На графиках наблюдаем линейную зависимость $\Delta P$ от $B^2$, как и ожидалось из теории.

Рассчитаем значения магнитной восприимчивости $\chi$:

$$\chi_{Cu} = -(1.08 \pm 0.03) \cdot 10^{-5} $$
$$\chi_{Al} = +(2.17 \pm 0.03) \cdot 10^{-5} $$

На рисунке ниже приведены зависимости перегрузки от квадрата индукции магнитного поля $P(B^2)$ для разных образцов графита.

\begin{figure}[H]
	\includegraphics[]{../gen/gr.pdf}
	\caption{Зависимость перегрузки $\Delta P$ от поля в зазоре $B$ для графита}
\end{figure}

\begin{table}[h]
	\caption{Параметры графика $\Delta P(B^2)$ для целого графита}
	\input{../gen/gr_mnk.tex}
\end{table}

$$\chi_{Gr} = +(2.15 \pm 0.11) \cdot 10^{-4} $$

На рисунке ниже приведены зависимости перегрузки от квадрата индукции магнитного поля $P(B^2)$ для вольфрама.

\begin{figure}[H]
	\includegraphics[]{../gen/wr_dbl.pdf}
	\caption{Зависимость перегрузки $\Delta P$ от поля в зазоре $B$ для вольфрама}
\end{figure}

\begin{table}[h]
	\caption{Параметры графика $\Delta P(B^2)$ для сдвоенного образца вольфрама}
	\input{../gen/wr_dbl_mnk.tex}
\end{table}

$$\chi_{W} = +(7.24 \pm 0.26) \cdot 10^{-5} $$
