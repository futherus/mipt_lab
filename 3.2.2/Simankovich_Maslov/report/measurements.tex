\section*{Экспериментальные результаты}

Для контуров с различными ёмкостями $C_n$, меняя их с помощью переключателя на блоке, измерим резонансные частоты $f_0$ и напряжения $U_C$ при установленном в напряжении $\mathscr{E}$ на выходе генератора. Для каждого значения $C_n$ по данным эксперимента проведем расчёт параметров стенда (см. таблицу \ref{tab:params}).

Рассчитаем средние значения $L$ и $R_L$ и их случайные погрешности для использования в дальнейшем (см. таблицу \ref{tab:RL}).

\begin{table}[H]
	\centering
	\footnotesize
	\input{../gen/setup-1.tex}
	\caption{Параметры контура}
	\label{tab:params}
\end{table}

\begin{table}[H]
	\centering
	\footnotesize
	\input{../gen/setup-mean-1.tex}
	\caption{Погрешности параметров}
	\label{tab:RL}
\end{table}

Построим графики амплитудно-частотные характеристик $U_C(f)$ для выбранных контуров (см. рис. \ref{fig:resonance}).

\begin{figure}[h]
	\centering
	\includegraphics[scale = 0.8]{../gen/fig-resonance.pdf}
	\caption{Графики амплитудно-частотные характеристик}
	\label{fig:resonance}
\end{figure}

\clearpage

Для контуров с двумя разными ёмкостями измерим амплитудно-частотные $U_C(f)$ и  фазово-частотные $\varphi(f)$ характеристики (см. таблицу \ref{tab:norm}).


Построим на одном графике амплитудно-частотные характеристики в безразмерных координатах $x = \frac{f}{f_0}$, $y = \frac{U_C}{U_C(f_0)}$. По ширине резонансных кривых на уровне 0,707
определим добротности $\mathcal{Q}$ соответствующих контуров:

$$\mathcal{Q}\left(33.2 \text{ нФ}\right) = 22.6 \pm 0.5$$
$$\mathcal{Q}\left(68 \text{ нФ}\right) = 16.6 \pm 0.3$$

\begin{figure}[H]
	\centering
	\includegraphics[scale = 0.8]{../gen/fig-resonance-norm.pdf}
	\caption{Графики амплитудно-частотные характеристик в нормированных координатах}
	\label{fig:norm}
\end{figure}


Построим на одном графике фазово-частотные характеристики в безразмерных координатах $x = \frac{f}{f_0}$, $y = \frac{\varphi}{\pi}$. По этим характеристикам определим добротности контуров по расстоянию между точками по оси $x$, в которых $y$ меняется от $1/4$ до $3/4$:

$$\mathcal{Q}\left(33.2 \text{ нФ}\right) = 20.7 \pm 0.4$$
$$\mathcal{Q}\left(68 \text{ нФ}\right) = 14.5 \pm 0.2$$


\begin{table}[H]
	\centering
	\footnotesize
	\caption{Погрешности параметров}
	\input{../gen/measure.tex}
	\label{tab:norm}
\end{table}



\begin{figure}[H]
	\centering
	\includegraphics[scale = 0.8]{../gen/fig-phase.pdf}
	\caption{Пробная катушка и ее положение относительно магнита}
\end{figure}
